%%%%%%%%%%%%%%%%%%%%%%%%%%%%%%
%
% $Autor: Wings $
% $Datum: 2020-01-24 14:19:22Z $
% $Pfad: komponenten/Bilderkennung/Produktspezifikation/JetsonNano/Allgemein/commands.tex $
% $Version: 1782 $
%
%
%%%%%%%%%%%%%%%%%%%%%%%%%%%%%%




%\titleformat{\chapter}{\bfseries\Huge}{\thechapter.\quad}{0em}{}

\renewcommand{\indexname}{Stichwortverzeichnis}


\DeclareMathOperator{\Atan2}{Atan2}
\DeclareMathOperator{\sign}{sign}
\DeclareMathOperator{\ReLU}{ReLU}

%Zahlenmengen
\newcommand{\C}{\mathbb{C}}
\newcommand{\R}{\mathbb{R}}
\newcommand{\N}{\mathbb{N}}
\newcommand{\Z}{\mathbb{Z}}
\newcommand{\Q}{\mathbb{Q}}
\newcommand{\Po}{\mathbb{P}}
\newcommand{\Rp}[2]{\mathbb{R}^{[#1;\,#2]}} % Menge der auf [a,b]-periodischen Funktionrn
%\DeclareMathOperator{\arg}{arg}

\newcommand*\justify{%
	\fontdimen2\font=0.4em% interword space
	\fontdimen3\font=0.2em% interword stretch
	\fontdimen4\font=0.1em% interword shrink
	\fontdimen7\font=0.1em% extra space
	\hyphenchar\font=`\-% allowing hyphenation
}


\newcommand{\MyComplex}[1]{\mathbf{#1}}
\newcommand{\Laplaceinv}[1]{#1}
\newcommand{\MapleCommand}[1]{\textcolor{MapleColor}{\texttt{\justify#1}}}
\newcommand{\PYTHON}[1]{\textcolor{PythonColor}{\texttt{\justify#1}}}
\newcommand{\SHELL}[1]{\textcolor{ShellColor}{\texttt{\justify#1}}}
\newcommand{\FILE}[1]{\textcolor{FileColor}{\texttt{\justify#1}}}
\newcommand{\Bez}{Bézier}
\newcommand{\PH}{pythagoreische Hodographen \xspace}
\newcommand{\PHd}{pythagoreischen Hodographen \xspace}





\setlength{\parindent}{0pt}




\definecolor{uuuuuu}{rgb}{0.26666666666666666,0.2666666666666666,0.26666666666666666}
\definecolor{qqqqff}{rgb}{0.,0.,1.}
\definecolor{MapleColor}{rgb}{1,0.5,0.}
\definecolor{PythonColor}{rgb}{0,0.5,1.}
\definecolor{ShellColor}{rgb}{1,0,0.5}
\definecolor{FileColor}{rgb}{0.5,0.5,1.}

\definecolor{LightGoldenrod}{rgb}{0.8,.9,0.3} 
\definecolor{AliceBlue}{rgb}{0.5,.8,1}
\definecolor{LightGrey}{rgb}{0.9,0.9,0.9} 
\definecolor{Beige}{rgb}{0.9,0.5,0.0} 
\definecolor{Gelb}{rgb}{0.999,0.999,0.0} 


\definecolor{LightCyan}{rgb}{0.88,1,1}
\definecolor{frenchblue}{rgb}{0.0, 0.45, 0.73}
\definecolor{greenblue}{rgb}{0.0, 0.25, 0.3}
\definecolor{darkcyan}{rgb}{0.0, 0.55, 0.55}
\definecolor{bondiblue}{rgb}{0.0, 0.58, 0.71}
\definecolor{grayleft}{rgb}{0.1, 0.1, 0.1}
\definecolor{grayright}{rgb}{0.2, 0.2, 0.2}
\definecolor{graycircle}{rgb}{0.3, 0.3, 0.3}
\definecolor{graylight}{rgb}{0.8, 0.8, 0.8}
\definecolor{greenenglish}{rgb}{0.0, 0.5, 0.0}
\definecolor{darkpastelgreen}{rgb}{0.01, 0.75, 0.24}
\definecolor{copper}{rgb}{0.72, 0.45, 0.2}
\definecolor{greenyellow}{rgb}{0.68, 1.0, 0.18}
\definecolor{fuchsia}{rgb}{1.0, 0.0, 1.0}
\definecolor{silver}{rgb}{0.75, 0.75, 0.75}
\definecolor{deepskyblue}{rgb}{0.0, 0.75, 1.0}


\newcommand{\GND}{\cellcolor{black}\textcolor{white}{GND}}
\newcommand{\Vf}{\cellcolor{red}\textcolor{black}{5V}}
\newcommand{\Vd}{\cellcolor{red}\textcolor{black}{3.3V}}
\definecolor{LightCyan}{rgb}{0.88,1,1}

\newcolumntype{a}{>{\columncolor{LightCyan}}c}


\newcounter{FortlaufendeNummer}
\setcounter{FortlaufendeNummer}{1}
\newcounter{LetztesKapitel}
\setcounter{LetztesKapitel}{-1}

\newcounter{BeamerChapter}
\setcounter{BeamerChapter}{1}


\graphicspath{%
	{../Bilder/},{Bilder/},{../../Bilder/},%
	{../../../../Allgemeines/},{../../../Allgemeines/},{../../Allgemeines/},%
	{../../../../Aufgaben/},{../../../Aufgaben/},%
	{../}}

\newcommand{\GRAPHICSC}[3]{\begin{center}
		\includegraphics[scale=#1]{#3}
\end{center}}

\newcommand{\GRAPHICS}[3]{\includegraphics[scale=#1]{#3}}

\newcommand{\FONO}{   
%\ifthenelse{\value{chapter} > \value{LetztesKapitel}}%
%{%
%\setcounter{LetztesKapitel}{\value{chapter}}%
%\setcounter{FortlaufendeNummer}{1}%
%}%
%\arabic{chapter}.\arabic{FortlaufendeNummer}.%
%\stepcounter{FortlaufendeNummer}%
}

\newcommand{\FONOBEAMER}{%
\ifthenelse{\value{BeamerChapter}>0}%
{%
  \ifthenelse{\value{BeamerChapter}>\value{LetztesKapitel}}%
  {%
    \setcounter{LetztesKapitel}{\value{BeamerChapter}}%
    \setcounter{FortlaufendeNummer}{1}%
}%
\hbox{}\arabic{BeamerChapter}.\arabic{FortlaufendeNummer}.\stepcounter{FortlaufendeNummer}%
}%
\hbox{}%
} 


\newcommand{\FONOBEAMERSTEP}{%
\ifthenelse{\value{BeamerChapter}>0}%
{%
  \ifthenelse{\value{BeamerChapter}>\value{LetztesKapitel}}%
  {%
    \setcounter{LetztesKapitel}{\value{BeamerChapter}}%
    \setcounter{FortlaufendeNummer}{1}%
}%
\stepcounter{FortlaufendeNummer}%
}%
\hbox{}%
\bigskip
} 

\newcommand{\FONOBEAMERSTAY}{   
\ifthenelse{\value{BeamerChapter}>\value{LetztesKapitel}}
{
\setcounter{LetztesKapitel}{\value{BeamerChapter}}
\setcounter{FortlaufendeNummer}{1}
}
\hbox{}\arabic{BeamerChapter}.\arabic{FortlaufendeNummer}.} 


% \FONOBEAMERSTEP
% \SATZNAME{}{}
% \SATZ{}
% \SATZNAMES{}{}
% \SATZS{}
% \STANDARD{}{}
% \STANDARDN{}{}  handout:0
% \STANDARDV{}{}  verbatim
% \DEF{}
% \DEFNAME{}{}
% \DEFS{}
% \DEFNAMES{}{}
% \BEMERKUNG{}
% \BEMERKUNGNAME{}{}
% \BEMERKUNGNAMES{}{}
% \BEISPIELNAME{}{}
% \BEISPIELNAMES{}{}
% \BEISPIEL{}
% \BEISPIELS{}



\newcommand{\Def}[1]
{
   \definecolor{shadecolor}{rgb}{0.98, 0.91, 0.71}

   \begin{snugshade*}
         \textbf{\FONOBEAMER  Definition.} #1
    \end{snugshade*}

    \bigskip
}

\newcommand{\DefN}[2]
{
   \definecolor{shadecolor}{rgb}{0.98, 0.91, 0.71}

   \begin{snugshade*}
         \textbf{ \FONOBEAMER Definition #1.}  #2
    \end{snugshade*}

    \bigskip
}


\newcommand{\Satz}[1]
{
   \definecolor{shadecolor}{rgb}{0.74, 0.83, 0.9}

   \begin{snugshade*}
         \textbf{\FONOBEAMER  Satz.} #1
    \end{snugshade*}

    \bigskip
}

\newcommand{\SatzN}[2]
{
   \definecolor{shadecolor}{rgb}{0.74, 0.83, 0.9}

   \begin{snugshade*}
         \textbf{ \FONOBEAMER Satz #1.}  #2
    \end{snugshade*}

    \bigskip
}

\newcommand{\Bemerkung}[1]
{
   \definecolor{shadecolor}{rgb}{0.66, 0.89, 0.63}

   \begin{snugshade*}
         \textbf{\FONOBEAMER  Bemerkung.} #1
    \end{snugshade*}

    \bigskip
}

\newcommand{\BemerkungN}[2]
{
   \definecolor{shadecolor}{rgb}{0.66, 0.89, 0.63} %Lavendel

   \begin{snugshade*}
         \textbf{ \FONOBEAMER Bemerkung #1.}  #2
    \end{snugshade*}

    \bigskip
}


\newcommand{\Beispiel}[1]
{
   \definecolor{shadecolor}{rgb}{0.9, 0.9, 0.98} 

   \begin{snugshade*}
         \textbf{\FONOBEAMER  Beispiel.} #1
    \end{snugshade*}

    \bigskip
}

\newcommand{\BeispielN}[2]
{
   \definecolor{shadecolor}{rgb}{0.9, 0.9, 0.98} 

   \begin{snugshade*}
         \textbf{ \FONOBEAMER Beispiel #1.}  #2
    \end{snugshade*}

    \bigskip
}


%todo Die Kommandos sind für das Endprodukt zu entfernen. Die entsprechenden Stellen sind zu bearbeiten bzw. zu löschen
\newcommand{\Mynote}[1]{\marginnote{\textcolor{red}{WS:#1}}}
\newcommand{\Ausblenden}[1]{}
\newcommand{\ToDo}[1]{\textcolor{red}{\section{ToDo} #1}}


\newcommand{\GRAFIKEINFUEGEN}{\textcolor{red}%
{\bf Hier ist eine Grafik einzufügen}}
\newcommand{\QUELLE}{\textcolor{red}{\bf Hier ist eine Quelle einzufügen}}
\newcommand{\neu}{\textcolor{red}{Wort Einfügen}}
\newcommand{\REFERENZ}{\textcolor{red}%
{\bf Referenz  einzufügen}}
\newcommand{\EXAMPLE}{\textcolor{red}%
{\bf Beispiel einzufügen}}


%
%\newcommand{\STANDARD}[1]{#1}
%\newcommand{\STANDARDN}[1]{#1}
%\newcommand{\beamergotobutton}[1]{#1}
%\newcommand{\MysectionZ}[2]{#2}
%
\newcommand{\trinom}[3]{\left(\begin{array}{c} #1\\#2\\#3 \\ \end{array}\right)}


\newcommand{\textdirectcurrent}{%
	\settowidth{\dimen0}{$=$}%
	\vbox to .85ex {\offinterlineskip
		\hbox to \dimen0{\leaders\hrule\hfill}
		\vskip.35ex
		\hbox to \dimen0{%
			\leaders\hrule\hskip.2\dimen0\hfill
			\leaders\hrule\hskip.2\dimen0\hfill
			\leaders\hrule\hskip.2\dimen0
		}
		\vfill
	}%
}
\newcommand{\mathdirectcurrent}{\mathrel{\textdirectcurrent}}

\newlist{notes}{enumerate}{1}
\setlist[notes]{label=Note: ,leftmargin=*}


% Kommandos für In-Line Code
\newcommand{\COMMAND}[1]{\lstinline[language=MyBash, style=inlinestyle]{#1}}
\newcommand{\mytt}[1]{\texttt{\footnotesize #1}}
\newcommand{\python}[1]{\lstinline[language=MyPython, style=inlinestyle]{#1}}
\newcommand{\twod}[2]{
    \ensuremath{{#1} \times {#2}}}
\newcommand{\threed}[3]{
    \ensuremath{{#1} \times {#2} \times {#3}}}

% Fett-geschriebene Tabellenberschriften
\renewcommand\theadfont{\bfseries}


\newcommand{\us}{\si{\micro\second}}
\newcommand{\usn}[1]{\SI{#1}{\micro\second}}
\renewcommand{\textapprox}{\raisebox{0.5ex}{\texttildelow}}

% Auswahl der Sprache
% 1.Argument ist der Pfad ohne "en" oder "de"
% 2.Argument ist der Dateiname
\newcommand{\InputLanguage}[2]{
\ifdefined\isGerman
  \input{#1de/#2}
\else
  \ifdefined\isEnglish
   \input{#1en/#2}
  \else
	\input{#1de/#2}
  \fi
\fi
}

\newcommand{\TRANS}[2]{%
	\ifdefined\isGerman	
	  #1	
	\else 
	  \ifdefined\isEnglish 
	    #2 
	  \else	
	    #1 
	  \fi 
	\fi
}
\renewcommand{\TRANS}[2]{#1}



\DeclareCaptionType{code}[Listing][Liste des Listings] 