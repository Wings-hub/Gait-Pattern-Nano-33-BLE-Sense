%%%
%
% $Autor: Wings $
% $Datum: 2021-05-14 $
% $Pfad: GitLab/MLEdgeComputer $
% $Dateiname: PyCharm
% $Version: 4620 $
%
% !TeX spellcheck = de_DE
% !TeX program = pdflatex
% !BIB program = biber
% !TeX encoding = utf8
%
%%%


\chapter{PyCharm}


\section{Konfigurieren einer virtuellen Umgebung}

\url{https://www.jetbrains.com/help/pycharm/creating-virtual-environment.html#python_create_virtual_env}


PyCharm ermöglicht die Verwendung des Virtualenv-Tools zum Erstellen einer projektspezifischen isolierten virtuellen Umgebung. Der Hauptzweck virtueller Umgebungen besteht darin, Einstellungen und Abhängigkeiten eines bestimmten Projekts unabhängig von anderen Python-Projekten zu verwalten. virtualenv Tool kommt mit PyCharm gebündelt, so dass der Benutzer nicht brauchen, um es zu installieren.

Seit der Version Python 3.3 ist das Modul \PYTHON{venv} integriert.

\subsection{Erstellen einer virtuellen Umgebung}
 
\begin{enumerate} 
Stellen Sie sicher, dass Sie Python auf Ihren Computer heruntergeladen und installiert haben.

Tun Sie eine der folgenden Optionen:

Klicken Sie auf den Python-Interpreter-Selektor, und wählen Sie Interpreter hinzufügen aus.

Drücken Sie , um die Projekteinstellungen/-einstellungen zu öffnen, und wechseln Sie zu Project . Klicken Sie The Configure project interpreter dann auf das Symbol und wählen Sie Hinzufügenaus.Ctrl+Alt+S



\end{enumerate}

