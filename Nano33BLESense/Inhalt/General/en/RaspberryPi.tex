%%%%%%%%%%%%%%%%%%%%%%%%
%
% $Autor: Wings $
% $Datum: 2019-07-09 09:26:07Z $
% $Pfad: Vorlesungen/WS_19_20/Projekte/KandaNeuralnetwork/latex - Ausarbeitung/Kapitel/Hardware.tex $
% $Version: 4440 $
%
%%%%%%%%%%%%%%%%%%%%%%%%





\section{Raspberry Pi}

The Raspberry Pi 3 B+ is the 85mmx56mmx17mm sized computer with an updated 64-bit quad core processor running at 1.4 GHz to enable fast computation .
 It has 5HZ  wireless LAN. It works with  Raspbian or PIXEL software. The Raspberry Pi has a Mobile Industry Processor Interface (MIPI) Camera Serial Interface Type 2 (CSI-2), which facilitates the connection of a small camera to the main Broadcom BCM2835 processor. This is a camera port providing an electrical bus connection between the two devices.
The USB 2.0 port on the Raspberry pi 3 B+ is used to connect to the Coral USB acclerator. The Raspberry Pi is equipped with a 40-pin header, see \autoref{fig:pinDiagram} for each individual pin's usage area. \href{https://www.raspberrypi.org/products/raspberry-pi-3-model-b-plus/}{\color{blue}raspberrypi.org}
\clearpage
\textbf{Raspberry Pi 3 B+ Specifications} \href{https://www.raspberrypi.org/magpi/raspberry-pi-3-specs-benchmarks/}{raspberry pi3.org/magpi}
%\centering
\begin{center}
 \begin{tabular}{
|c| c|} 
 \hline
 Specifications &Raspberry Pi 3B+\\
 \hline\hline
  CPU type/speed&4xARM Cortex-A53,1.2GHz \\ 
 \hline
 RAM size& 1GB LPDDR2 (900 MHz)\\
 \hline
 GPIO& 40-pin header, populated\\
 \hline
 Networking &10/100 Ethernet, 2.4GHz 802.11n wireless\\
 \hline
 Storage&microSD\\
 \hline
 Bluetooth & Bluetooth 4.1 Classic, Bluetooth Low Energy\\
 \hline
 \end{tabular}
\end{center}

\begin{figure}[!h]
\centering
\includegraphics[width=0.5\textwidth]{RaspberryPi/raspberrypi} 
\caption{Raspberrypi; source: \href{https://www.raspberrypi.org/products/raspberry-pi-3-model-b-plus/}{raspberry-pi-3B+} }
\end{figure}
\begin{figure}[!h]
\centering
\includegraphics[width=0.8\textwidth]{RaspberryPi/pindiagramrpi}
\caption{Pindiagram; source: \href {https://www.mobilefish.com/images/developer/raspberry\_pi3\_model\_b\_pin\_diagram.jpg}{raspberry pi\_3b}}
\label{fig:pinDiagram}
\end{figure}
\Mynote{Replace figure?}
\clearpage

