%%%%%%%%%%%%%%%%%%%%
%
% $Autor: Wings $
% $Datum: 2020-02-24 14:29:03Z $
% $Pfad: komponenten/Bilderkennung/Produktspezifikation/CorelTPU/Ausarbeitung/Kapitel/Software.tex $
% $Version: 1791 $
%
%
%%%%%%%%%%%%%%%%%%%





\section{Einrichten der Kamera}

Nach dem Systemstart kann das  Menü Raspberry Pi 
Symbol oben links aktiviert werden. Dort kann man in den Einstellungen \menu[>]{Preferences > Raspberry Pi Konfiguration > Interfaces}. Zur Verwendung der muss dann die Kamera aktiviert (Enabled) werden. Nach der Aktivierung der Kamera muss der Rechner neugestartet werden. Ein Dialog mit einer Kammandozeile kann mittels \keys{\ctrl + \Alt + T} geöffnet werden. Dort kann man durch folgende Eingaben eine Testaufnahme eines Bildes machen:

\begin{lstlisting}
raspistill -v -o test.jpg
xdg-open /home/pi/test.jpg
\end{lstlisting}

Das Bild wird dann in der Datei test.jpg abgelegt.


%\begin{notes}
%\item Das Pi-Kameraprogramm des Kameramoduls hilft bei der Durchführung der Echtzeit-Live-Identifikation von Objekten.
%Damit das Pi-Kamera-Modul speziell für die Live-Objektidentifizierung im VNC-Viewer funktioniert, aktivieren Sie den Direktaufnahmemodus des VNC-Viewers.
%\end{notes}
