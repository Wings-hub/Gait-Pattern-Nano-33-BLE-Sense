%%%%%%
%
% $Autor: Bakke $
% $Datum: 2021-08-02 15:57:45Z $
% $Pfad: WuSt/Skript/Produktspezifikation/powerpoint/ImageProcessing.tex $
% $Version: 4620 $
%
%%%%%%

\chapter{Setting Up the USB Accelerator}
\label{chap:Setup}

For these next steps, disconnect the USB Accelerator from the Raspberry Pi by pulling out the USB-C connector.

\section{Install the Edge TPU Runtime and PyCoral Library}

The first step is to add Coral's Debian package repository to the system. To do so, enter these lines in the terminal:



\begin{verbatim}
	echo "deb https://packages.cloud.google.com/apt coral-edgetpu-stable main" | sudo tee /etc/apt/sources.list.d/coral-edgetpu.list
	
	curl https://packages.cloud.google.com/apt/doc/apt-key.gpg | sudo apt-key add -
	
	sudo apt-get update
\end{verbatim}


The next step is to install the PyCoral library with the following command:

\begin{verbatim}
	sudo apt-get install python3-pycoral
\end{verbatim}

\begin{notes}
	\item This command also installs the library libedgetpu1-std automatically.
\end{notes}


 Now connect the USB Accelerator to your computer using the provided USB 3.0 cable. If you already plugged it in, remove it and replug it so the newly-installed udev rule can take effect.

\subsection{TensorFlow Lite Runtime}

If TensorFlow is not installed on the device, a simplified package called \FILE{tflite$\_$runtime} containing just the TensorFlow Lite interpreter can be installed, which is all that's required for this step. Enter the following commands in the terminal to install the package:

\begin{verbatim}
	echo "deb https://packages.cloud.google.com/apt coral-edgetpu-stable main" | sudo tee /etc/apt/sources.list.d/coral-edgetpu.list
	
	curl https://packages.cloud.google.com/apt/doc/apt-key.gpg | sudo apt-key add -
	
	sudo apt-get update
	
	sudo apt-get install python3-tflite-runtime
\end{verbatim}

\section{Testing Object Classification and Object Detection}

\subsection{Setup}

It is recommended to create a directory for the projects

\begin{verbatim}
	mkdir Projects_CoralTPU
\end{verbatim}

\subsubsection{Step 1: Clone Repository}


\begin{verbatim}
	cd /home/pi/Projects_CoralTPU
	
	mkdir google-coral && cd google-coral
	
	git clone https://github.com/google-coral/examples-camera --depth 1
\end{verbatim}


\subsubsection{Step 2: Download Models}

\begin{verbatim}
	cd examples-camera
	
	sh download_models.sh
\end{verbatim}


\subsubsection{Step 3: Install the GStreamer Libraries}

\begin{verbatim}
	cd gstreamer
	
	bash install_requirements.sh
\end{verbatim}

\subsection{Run the Demos}

\subsubsection{classify.py}

\begin{verbatim}
	python3 classify.py
\end{verbatim}


By default, this uses the \FILE{mobilenet$\_$v2$\_$1.0$\_$224$\_$quant$\_$edgetpu.tflite} model.

You can change the model and the labels file using flags \FILE{---model} and \FILE{---labels.}

\subsubsection{detect.py}

\begin{verbatim}
	python3 detect.py
\end{verbatim}

Likewise, you can change the model and the labels file using \FILE{---model} and \FILE{---labels}.

\clearpage