%%%%%%%%%%%%%%%%%%%%%%%%
%
% $Autor: Wings $
% $Datum: 2019-07-09 09:26:07Z $
% $Pfad: Vorlesungen/WS_19_20/Projekte/KandaNeuralnetwork/latex - Ausarbeitung/Kapitel/Hardware.tex $
% $Version: 4440 $
%
%%%%%%%%%%%%%%%%%%%%%%%%





\section{Raspberry Pi}

Der Raspberry Pi 3 B+ ist ein 85 mm x 56 mm x 17 mm großer Computer, welcher mit einem aktuellen 64-Bit-Quadcore-Prozessor ausgestattet ist. Um Berechnungen schnell durchführen zu können, wird dieser mit einer Taktfrequenz von 1,4 GHz betrieben.
Damit dieser mit dem Internet kommunizieren kann, ist zudem ist ein 5 GHz WLAN-Funkchip integriert.
Das Betriebssystem des Raspberry Pi wird als Raspbian oder Raspberry Pi OS bezeichnet.
Das Mobile Industry Processor Interface (MIPI) sowie das Camera Serial Interface Type 2 (CSI-2) ermöglicht den Anschluss einer kleinen Kamera an den Broadcom-Hauptprozessor BCM2835. Dies ist ein Kameraanschluss, der eine elektrische Feldbusverbindung zwischen den beiden Geräten herstellt. Der USB 2.0-Anschluss des Raspberry pi 3 B+ wird für den Anschluss des Coral USB-Acclerators verwendet. 
\href{https://www.raspberrypi.org/products/raspberry-pi-3-model-b-plus/}{raspberrypi.org}
\clearpage
\textbf{Raspberry Pi 3 B+ Specifications}\href{https://www.raspberrypi.org/magpi/raspberry-pi-3-specs-benchmarks/}{raspberry pi3.org/magpi}
%\centering
\begin{center}
 \begin{tabular}{
|c| c|} 
 \hline
 Specifications &Raspberry Pi 3B+\\
 \hline\hline
  CPU type/speed&4xARM Cortex-A53,1.2GHz \\ 
 \hline
 RAM size& 1GB LPDDR2 (900 MHz)\\
 \hline
 GPIO& 40-pin header, populated\\
 \hline
 Networking &10/100 Ethernet, 2.4GHz 802.11n wireless\\
 \hline
 Storage&microSD\\
 \hline
 Bluetooth & Bluetooth 4.1 Classic, Bluetooth Low Energy\\
 \hline
 \end{tabular}
\end{center}

\begin{figure}[!h]
\centering
\includegraphics[width=0.5\textwidth]{RaspberryPi/raspberrypi} 
\caption{Raspberrypi; source:\href{https://www.raspberrypi.org/products/raspberry-pi-3-model-b-plus/}{raspberry-pi-3B+} }
\end{figure}
\begin{figure}[!h]
\centering
\includegraphics[width=0.5\textwidth]{RaspberryPi/pindiagramrpi}
\caption{Pindaiagram;source:\href {https://www.mobilefish.com/images/developer/raspberry\_pi3\_model\_b\_pin\_diagram.jpg}{raspberry pi\_3b}}
\end{figure}
\clearpage

