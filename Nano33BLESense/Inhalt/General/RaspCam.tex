%%%%%%
%
% $Autor: Wings $
% $Datum: 2021-05-14 $
% $Pfad: GitLab/MLEdgeComputer $
% $Dateiname: RaspCam
% $Version: 4620 $
%
% !TeX spellcheck = de_DE
%
%%%%%%



\chapter{RASP CAM HQ Raspberry Pi - Kamera, 12MP, Sony IMX477R}

\begin{itemize}
  \item \url{https://www.arducam.com/raspberry-pi-high-quality-camera-lens/}
  \item \url{https://www.raspberrypi.org/products/raspberry-pi-high-quality-camera/0}
  \item \url{https://www.reichelt.de/de/de/raspberry-pi-kamera-12mp-sony-imx477r-rasp-cam-hq-p276919.html?PR}
\end{itemize}



RASP CAM HQ

Die Raspberry Pi High Quality Kamera ist das neueste Kamerazubehör von Raspberry Pi (ohne Objektiv, siehe Zubehör).


Sie bietet eine höhere Auflösung zum Vorgängermodell v2 (12 Megapixel, im Vergleich zu 8 Megapixeln), eine höhere Empfindlichkeit (ca. 50 \% mehr Fläche pro Pixel für eine verbesserte Leistung bei schlechten Lichtverhältnissen) und ist für den Einsatz mit Wechselobjektiven in C- und CS-Fassung ausgelegt. Andere Objektivformfaktoren können jedoch mit Objektivadaptern von Drittanbietern angepasst werden.

Sie ist unter anderem für Industrie- und Verbraucheranwendungen, einschließlich Sicherheitskameras konzipiert worden, die ein Höchstmaß an visueller Wiedergabetreue und/oder die Integration mit Spezialoptiken erfordern.

\bigskip

Technische Daten


\begin{itemize}
  \item kompatibel mit allen Raspberry Pi Modellen
  \item Sony IMX477R Sensor
  \item 12,3 Megapixel Auflösung
  \item 7,9 mm Sensor-Diagonale
  \item $1,55 \mu m \times  1,55 \mu m$ Pixelgröße
  \item Ausgabe: ROH12/10/8, COMP8
  \item Fokus zurück: Einstellbar (12,5 mm - 22,4 mm)
  \item Objektiv-Standards: C-Mount
  \item integrierter IR-Sperrfilter
  \item Stativ-Montage: 1/4\grqq{}-20
\end{itemize}

Lieferumfang


\begin{itemize}
  \item Leiterplatte mit einem Sony IMX477-Sensor (ohne Objektiv, siehe Zubehör)
  \item ein 200 mm FPC-Kabel für den Anschluss an einem Raspberry PI
  \item eine gefräste Aluminium-Objektivfassung mit integriertem Stativanschluss und Fokuseinstellring
  \item ein C- auf CS-Mount-Adapter
\end{itemize}

\bigskip



Optional erhältlich (siehe Zubehör)

\begin{itemize}
  \item 6 mm Weitwinkel-Objektiv mit der Artikelnr. RPIZ CAM 6MM WW
  \item 16 mm Teleobjektiv mit der Artikelnr. RPIZ CAM 16MM TO
\end{itemize}

\bigskip
     Hinweis

\begin{itemize}
  \item zum Betreiben der Kamera wird die neuste Raspberry Pi Software benötigt
  \item Produktion bis mindestens Januar 2026
\end{itemize}


\begin{tabular}{ll}
Ausführung & \\
Modell &  Raspberry Pi \\
Video \@ 30 fps &  $1920 \times 1080$ Pixel  \\
Foto  &  $4056 \times 3040$ Pixel \\
Allgemeines& \\
Ausführung  & Standard  \\
Auflösung   &12 MP  \\
Video \@ 60 fps  & 1280 x 720 Pixel \\
Betrachtungswinkel &  $75^\circ$ \\
Bildsensor  & 1/4\grqq \\
Anschlüsse / Schnittstellen & \\
Anschluss  & CSI  \\
Sonstiges & \\
Spezifikation  & IMX477R \\
Lampeneigenschaften & \\
Länge  & 38 mm  \\
Maße  &  \\
Breite  & 38 mm  \\
Herstellerangaben & \\
Hersteller &  RASPBERRY PI \\
Artikelnummer des Herstellers &  SC0261 \\
Verpackungsgewicht &  0.046 kg \\
RoHS  & konform \\
EAN / GTIN &  0633696492738\\
\end{tabular}

\bigskip

Produktinformationen \glqq Raspberry Pi High Quality Kamera\grqq 

Die Raspberry Pi High Quality Camera ist das neueste Kameramodul der Raspberry Pi Foundation. Sie bietet eine höhere Auflösung (12 Megapixel, im Vergleich zu 8 Megapixeln) und Empfindlichkeit (ca. 50 \% mehr Fläche pro Pixel für eine verbesserte Leistung bei schlechten Lichtverhältnissen) als das bestehende Kameramodul v2 und ist für den Einsatz mit Wechselobjektiven in C- und CS-Mount-Formaten ausgelegt. Andere Objektiv-Formfaktoren können mit Objektivadaptern von Drittanbietern verwendet werden.

Die High Quality-Kamera bietet eine Alternative zum Kameramodul v2 für Industrie- und Verbraucheranwendungen, einschließlich Sicherheitskameras, die ein Höchstmaß an visueller Wiedergabetreue und/oder Integration mit Spezialoptiken erfordern. Sie ist mit allen Modellen von Raspberry Pi Computer ab Raspberry Pi 1 Modell B kompatibel, wobei die neueste Software-Version von \url{www.raspberrypi.org}. verwendet wird.



Das Paket umfasst eine Platine mit einem Sony IMX477-Sensor, ein FPC-Kabel für den Anschluss an einen Raspberry Pi-Computer, eine gefräste Aluminium-Linsenfassung mit integrierter Stativhalterung und Fokuseinstellring sowie einen C- auf CS-Mount-Adapter. Mittels Adapterkabel kann die Kamera außerdem auch mit dem Raspberry Pi Zero verwendet werden.

\bigskip


Technische Daten

Sensor: 

Sony IMX477R Gestapelter, hintergrundbeleuchteter Sensor 

7,9 mm Sensor-Diagonale

$1,55 \mu m \times 1,55 \mu m$ Pixelgröße

Ausgang: RAW12/10/8, COMP8

Auflagemaß: Einstellbar (12,5 mm-22,4 mm)

\bigskip

Objektivtyp:

C-Mount

CS-Mount (C-CS Adapter inklusive)

IR-Sperrfilter: Integriert (Der Sperrfilter kann entfernt werden, dies ist jedoch irreversibel) 

Länge des Flachbandkabels: 200 mm

Stativanschluss: 1/4\grqq{}-20

Die Raspberry Pi High Quality Camera wird mindestens bis Januar 2026 in Produktion bleiben



\bigskip

\includegraphics[width=0.4\textwidth]{RaspCam/RaspCam} \quad \includegraphics[width=0.4\textwidth]{RaspCam/RaspCam2}

\includegraphics[width=0.4\textwidth]{RaspCam/RaspCam3} 
\quad
\includegraphics[width=0.4\textwidth]{RaspCam/RaspCam4}


MAXIMALE AUFLÖSUNG DER KAMERA
Technisch wäre der CCD Sensor der Raspberry Pi HQ Kamera, gemäß Datenblatt, in der Lage, Fotos und Videos mit einer Auflösung von 4K / 60Hz aufzunehmen.

Dies wäre aber nur mit einem 4-Lane CSI-2 Interface möglich.
Der Raspberry Pi nutzt jedoch ein 2-Lane CSI-2 Interface und hat darüber hinaus auch eine Limitierung durch den H.264 Hardware Encoder.

Somit ist die Auflösung der HQ Kamera in Verbindung mit dem Raspberry Pi auf 1920×1080 für Videoaufnahmen und 2592×1944 für Fotos begrenzt.

BRENNWEITE, WEITWINKEL, TELE, ZOOM – KLEINER OBJEKTIV-GUIDE
Bei der Auswahl eines passenden Objektivs für die HQ Kamera, kommt man unweigerlich mit Fachbegriffen wie Brennweite, Tele- oder Weitwinkelobjektiv in Berührung.

Hier eine kurze Erklärung:

Grundsätzlich gilt: Die Brennweite eines Objektivs bestimmt das Sichtfeld deiner Kamera und wird standardmäßig in Millimetern (mm) gemessen.
Als Faustregel kann man folgendes sagen: Je höher die Nummer / Zahl der Brennweite, desto enger das Sichtfeld und desto dichter / größer erscheint das zu fotografierende Objekt.

Das bedeutet bei einem Objektiv mit 6 mm Brennweite, dass du einen großen Bereich mit der Kamera “sehen” kannst, das Objekt was du fotografieren möchtest jedoch relativ klein erscheint.
Weitwinkelobjektive eignen sich daher ideal für Landschaftsaufnahmen oder Aufnahmen einer Gruppe von Personen.

Wenn du nun den Kirchturm in der Landschaft oder die Schwiegermutter in der Hochzeitsgesellschaft fotografieren möchtest, eignet sich ein Teleobjektiv mit einer Brennweite ab 16 mm.

Die ideale Symbiose aus Weitwinkel und Tele ist ein Zoomobjektiv. Hier lässt sich die Brennweite über einen Zoomring flexibel ändern und das zu fotografierende Objekt erscheint somit größer oder kleiner.

Die Firma Nikon hat auf dieser Seite einen super Objektivsimulatior, wo das Thema Brennweite nochmal in Verwendung mit einer Nikon Kamera verständlich gezeigt wird.

VERBINDUNG / ANSCHLUSS DER KOMPONENTEN
Da der Raspberry Pi Zero einen kleineren Kameraanschluss hat, musst du zunächst das bei der HQ-Kamera mitgelieferte Kabel austauschen.
Dazu musst du vorsichtig den schwarzen Bügel des ZIF Connectors nach hinten ziehen.
Achtung: Wende hier nicht zu viel Kraft an, da der Anschluss sehr empfindlich ist!

Ziehe nun das weiße Kabel aus dem Anschluss und verbinde das Raspberry Pi Zero Adapterkabel, in dem du das Kabel mit den goldenen Kontaktflächen der breiten Seite nach unten in den Connector schiebst. Anschließend schiebst du den schwarzen Bügel wieder vorsichtig nach vorne.

Die schmale Seite des Adapterkabels kommt anschließend ebenfalls mit den goldenen Kontakten nach unten zeigend in den Connector des Raspberry Pi Zero.

Schiebe nun noch die bespielte microSD Karte in den Pi Zero und verbinde das Micro USB Kabel mit der Buchse USB / J10.

Deine Raspberry Pi Webcam ist nun einsatzbereit!

VERBINDUNG MIT DEM COMPUTER UND ERSTER TEST
Du kannst nun den USB A Stecker des micro USB Kabels mit deinem Computer verbinden. Die Webcam wird nach ca. 10 Sekunden ohne Treiberinstallation als “Piwebcam” erkannt und steht dir in allen Programmen mit Webcam-Unterstützung als Kamera zur Auswahl.
\chapter{RPIZ CAM 16mm}

\url{https://www.reichelt.de/raspberry-pi-16mm-kameralinse-teleobjektiv-rpiz-cam-16mm-to-p276921.html?&nbc=1&trstct=lsbght_sldr::276919}



Dieses 16mm Teleobjektiv ist für die originale Raspberry Pi Kamera konzipiert worden



Lieferumfang

16 mm Teleobjektiv (ohne Bildsensor-Platine, siehe Zubehör)


\begin{tabular}{ll}
Ausführung &\\
Modell     &  Raspberry Pi \\
Allgemeines & \\
Ausführung  &  Zubehör \\
Sonstiges & \\
Spezifikation  & Tele \\
Herstellerangaben & \\
Hersteller &  RASPBERRY PI \\
Artikelnummer des Herstellers  & SC0123 \\
Verpackungsgewicht &  0.214 kg \\
RoHS &  konform \\
EAN / GTIN &  9900002769213 \\
\end{tabular}

\includegraphics[width=0.4\textwidth]{RaspCam/RpizCam16mmWW} \quad \includegraphics[width=0.4\textwidth]{RaspCam/RpizCam16mmWW2}

\includegraphics[width=0.4\textwidth]{RaspCam/RpizCam16mmWW3} \quad \includegraphics[width=0.4\textwidth]{RaspCam/RpizCam16mmWW4}


\chapter{RPIZ CAM 6mm WW}

\url{https://www.reichelt.de/raspberry-pi-6mm-kameralinse-weitwinkel-rpiz-cam-6mm-ww-p276922.html?&nbc=1&tr}


Dieses Weitwinkel-Kameraobjektiv ist für die originale Raspberry Pi Kamera konzipiert worden.

Lieferumfang

6 mm Weitwinkelobjektiv (ohne Bildsensor-Platine, siehe Zubehör)


Blende: F1.2

Gewinde: CS

minimaler Objektabstand: 20 cm



\begin{tabular}{ll}
Ausführung & \\
Modell &  Raspberry Pi \\
Allgemeines & \\
Ausführung &   Zubehör \\
Sonstiges & \\
Spezifikation  & Weitwinkel \\
Herstellerangaben & \\
Hersteller &   RASPBERRY PI \\
Artikelnummer des Herstellers  & SC0124 \\
Verpackungsgewicht &  0.107 kg \\
RoHS &  konform \\
EAN / GTIN &  9900002769220 \\
\end{tabular}

\includegraphics[width=0.4\textwidth]{RaspCam/RpizCam6mmWW} \quad \includegraphics[width=0.4\textwidth]{RaspCam/RpizCam6mmWW2}

\includegraphics[width=0.4\textwidth]{RaspCam/RpizCam6mmWW3} \quad \includegraphics[width=0.4\textwidth]{RaspCam/RpizCam6mmWW4}

\includegraphics[width=0.4\textwidth]{RaspCam/RpizCam6mmWW5} \quad \includegraphics[width=0.4\textwidth]{RaspCam/RpizCam6mmWW6}
